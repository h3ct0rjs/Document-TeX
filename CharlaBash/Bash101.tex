\documentclass{beamer}
%Bash Scripting Slides , This is part of the Research that I've been doing in UTP .
%feedback : hfjimenez@utp.edu.co
%

\mode<presentation>
{
  \usetheme{CambridgeUS}      % or try Darmstadt, Madrid, Warsaw, ...
  \usecolortheme{seagull} % or try albatross, beaver, crane, ...
  \usefonttheme{default}  % or try serif, structurebold, ...
  \setbeamertemplate{navigation symbols}{}
  \setbeamertemplate{caption}[numbered]
} 
%Language and Input encoding 
\usepackage[spanish]{babel}
\usepackage[utf8x]{inputenc}

\usepackage{listings}	%Resaltar Codigo 
\usepackage{color}		%Colores para resaltar codigo

\definecolor{mygreen}{rgb}{0,0.6,0} %Definicion de los tres colores 
\definecolor{mygray}{rgb}{0.5,0.5,0.5}
\definecolor{mymauve}{rgb}{0.58,0,0.82}

\lstset{ %
  backgroundcolor=\color{white},   % choose the background color
  basicstyle=\footnotesize,        % size of fonts used for the code
  breaklines=true,                 % automatic line breaking only at whitespace
  captionpos=b,                    % sets the caption-position to bottom
  commentstyle=\color{mygreen},    % comment style
  escapeinside={\%*}{*)},          % if you want to add LaTeX within your code
  keywordstyle=\color{blue},       % keyword style
  stringstyle=\color{mymauve},     % string literal style
}

	
\title[Bash Scripting 101]{Bash Scripting 101}
\author{H\'ector Fabio Jim\'enez S. @c1b3rh4ck}
\institute{Semillero de Investigaci\'on Pulpa \\ UTP}
\date{Enero de 2014}

\definecolor{mygreen}{rgb}{0,0.6,0}
\definecolor{mygray}{rgb}{0.5,0.5,0.5}
\definecolor{mymauve}{rgb}{0.58,0,0.82}

\lstset{ %
  backgroundcolor=\color{white},   % choose the background color
  basicstyle=\footnotesize,        % size of fonts used for the code
  breaklines=true,                 % automatic line breaking only at whitespace
  captionpos=b,                    % sets the caption-position to bottom
  commentstyle=\color{mygreen},    % comment style
  escapeinside={\%*}{*)},          % if you want to add LaTeX within your code
  keywordstyle=\color{blue},       % keyword style
  stringstyle=\color{mymauve},     % string literal style
}
%%%%%%%%%%%%%%%%%%%%%%%%%%%%%%%%%%%%%%%%%%%%%%%%%%%%%%%%%
%Documento Empieza Aqui!
%%%%%%%%%%%%%%%%%%%%%%%%%%%%%%%%%%%%%%%%%%%%%%%%%%%%%%%%%
\begin{document}

\begin{frame}
  \titlepage
\end{frame}

%%%%%%%%%%%%%%%%%%%%%%%%%%%%%%%%%%%%%%%%%%%%%%%%%%%%%%%%%
% Uncomment these lines for an automatically generated outline.
\begin{frame}{Agenda}
  \tableofcontents
\end{frame}

%%%%%%%%%%%%%%%%%%%%%%%%%%%%%%%%%%%%%%%%%%%%%%%%%%%%%%%%%
%Estructura de la charla :
%
%
%%%%%%%%%%%%%%%%%%%%%%%%%%%%%%%%%%%%%%%%%%%%%%%%%%%%%%%%%
\section{Introducci\'on}
\begin{frame}{Que es una shell?}

\begin{exampleblock}{}
  {\large ``Es un interprete de comandos que te permite introducir ordenes para comunicarse directamente con el nucleo.''}
  
    
\end{exampleblock}

%\vskip 1cm
%\begin{block}{Examples}
%Some examples of commonly used commands and features are included, to help you get started.
%\end{block}
\end{frame}
%%%%%%%%%%%%%%%%%%%%%%%%%%%%%%%%%%%%%%%%%%%%%%%%%%%%%%%%%

%%%%%%%%%%%%%%%%%%%%%%%%%%%%%%%%%%%%%%%%%%%%%%%%%%%%%%%%%
% Commands to include a figure:
%\begin{figure}
%\includegraphics[width=\textwidth]{your-figure's-file-name}
%\caption{\label{fig:your-figure}Caption goes here.}
%\end{figure}

\end{frame}
%%%%%%%%%%%%%%%%%%%%%%%%%%%%%%%%%%%%%%%%%%%%%%%%%%%%%%%%%
\section{Python Code }
\begin{frame}[fragile]
\frametitle{tEST}

\begin{lstlisting}[language=python]
>>> from numpy import *
>>> from numpy.fft import *
>>> signal = array([-2., 8., -6., 4., 1., 0., 3., 5.])
\end{lstlisting}
\end{frame}
%%%%%%%%%%%%%%%%%%%%%%%%%%%%%%%%%%%%%%%%%%%%%%%%%%%%%%%%%

%%%%%%%%%%%%%%%%%%%%%%%%%%%%%%%%%%%%%%%%%%%%%%%%%%%%%%%%%
\section{Contacto}
\begin{frame}{}

\center{\Large{hfjimenez@utp.edu.co \\
\vskip 1cm
 @c1b3rh4ck }}
 
 \section{Pausitas y Over}

\end{frame}
%EOF
\end{document}